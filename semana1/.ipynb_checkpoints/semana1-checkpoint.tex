\section{Semana 1}
La referencia de este capitulo esta en el siguiente paper \href{https://arxiv.org/pdf/1505.01394.pdf}{https://arxiv.org/pdf/1505.01394.pdf}
\subsection{Coherencia en Series de tiempo}
Para introduccir al concepto de coherencia, presentaremos un ejemplo en el contexto de series de tiempo. Entonces, sean las series $Z_1(t)$ y $Z_2 (t)$ dos series de tiempo débilmente estacionarias con función de convarianza con la siguiente estructura:
$$
\cov{
Z_k(t+h) ; Z_k(t)
} = C_{kk}(h)
$$
Y la cruzada con la siguiente estructura:
$$
\cov{
Z_k(t+h) ; Z_l(t)
} = C_{kl}(h)
$$
También se sabe que sus densidades espectrales vienen dadas por:
$$
f_{kl}(\omega) = \parcurvo{2\pi}^{-1/2}\int_{\R{}} C_{kl}(h)\exp\parllave{- i \omega h} dh
$$
\begin{defi}{Coherencia de Series de tiempo}
Se define la función de coherencia cuadratica entre dos series de tiempos dependiente de una frecuencia $\omega$ a la función:
$$
\gamma^2(\omega) = \dfrac{
\abs{f_{12}(\omega)}^2
}{
f_{11}(\omega) f_{22}(\omega)
}
$$
Esta función puede interpretarse como una cuantificación de la relación lineal entre dos series de tiempo a una frecuencia $\omega$.
\end{defi}
\subsection{Coherencia en campos aleatorios}
Sea el campo aleatorio complejo dado por $\bf{Z}\parcurvo{\bf{s}} = \parcurvo{ Z_1(\bf{s})}$